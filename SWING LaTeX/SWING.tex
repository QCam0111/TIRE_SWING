\documentclass[journal,twoside,web]{ieeecolor}
\usepackage{generic}
\usepackage{cite}
\usepackage{amsmath,amssymb,amsfonts}
\usepackage{algorithmic}
\usepackage{graphicx}
\usepackage{textcomp}
\usepackage{array}
\newcolumntype{L}[1]{>{\raggedright\let\newline\\\arraybackslash\hspace{0pt}}m{#1}}
\newcolumntype{C}[1]{>{\centering\let\newline\\\arraybackslash\hspace{0pt}}m{#1}}
\newcolumntype{R}[1]{>{\raggedleft\let\newline\\\arraybackslash\hspace{0pt}}m{#1}}
\def\BibTeX{{\rm B\kern-.05em{\sc i\kern-.025em b}\kern-.08em
    T\kern-.1667em\lower.7ex\hbox{E}\kern-.125emX}}
\markboth{\journalname, VOL. XX, NO. XX, XXXX 2023}
{Author \MakeLowercase{\textit{et al.}}: Short Wave Infrared Neuromodulation Gadget (May 2023)}
\begin{document}
\title{Short Wave Infrared Neuromodulation Gadget (May 2023)}
\author{Cameron R. Author, \IEEEmembership{Member, IEEE}, Matthew T. Author, Bibhus L. Author, 
        Krishna S. Sponsor, \IEEEmembership{Member, IEEE}, John L. Mentor, \IEEEmembership{Member, IEEE}
\thanks{ }
\thanks{ }
\thanks{ }
\thanks{ }}

\maketitle

\begin{abstract}
    Direct stimulation of neurons in the brain can potentially treat many diseases, such as Parkinson's  
    or Alzheimer's disease. Direct stimulation, whether it be through electric or photonic stimulation,  
    provided a way to activate neurons in the brain and treat diseases and conditions. However, this kind  
    of invasive stimulation can have risks that lead to worsening the condition or cause infection.  
    The Short-Wave Infrared Neuromodulation Gadget (SWING) aimed to build and test a non-invasive optical method  
    of stimulation with funding provided by the KIND Laboratory's Brain IMPACT project. SWING is part of the  
    two semester long Electrical and Computer Engineering capstone sequence at The Ohio State University. 
    
    Because the optical coefficients of biological tissue are not well known at 1550 nm,  
    SWING used a cubic extrapolation to approximate these coefficients. Monte Carlo eXtreme (MCX) was then used to predict the  
    expected photon distribution and intensity throughout a model of the human head. MCX was ran multiple times with different positions 
    and wavelengths using The Ohio State Supercomputer. MCX showed that deep brain stimulation is possible at all the wavelengths tested. 
    Based on the MCX results, 1550nm wavelength is the best choice for further testing. Solving the problems previously discussed has the 
    potential to reduce the effects of brain disorders on the general population, mitigate the risks of surgery that patients would have to 
    go through if done invasively, and to improve overall health. 
\end{abstract}

\begin{IEEEkeywords}

\end{IEEEkeywords}

\section{Introduction}
\label{sec:introduction}
There are many physical issues in the brain, including Parkinson's disease and functional problems such as attention-deficit/hyperactivity disorder (ADHD)  
and depressive disorders. A solution to such problems that has been explored recently in the Neurotech community is one that involves a direct stimulation of  
neuronal connections in the brain [1]. Direct neuronal stimulation, whether it be through electric or photonic stimulation, provides a way to control  
mechanisms in the brain and treat diseases and conditions. These treatments result in an improvement of the effects caused by these diseases and conditions.  
However, most modern neuromodulation strategies are invasive in nature, and there are limited options for a non-invasive approach to neuromodulation for medical benefit.  
Many invasive techniques involve surgical implants and increase the risk of brain hemorrhage and worsening mental and emotional status for some patients,  
that often make the cons worse for life-threatening conditions [1]. As a result, SWING looks to investigate a non-invasive method for neuromodulation  
using a near-infrared photonic stimulation method. 

\section{Methods}
\label{sec:methods}

\section{Results}
\label{sec:results}

Table ~\ref{CoefficientTable} below displays the estimated absorption and scattering coefficients for each of the biological tissue layers as well as each wavelength.
These values were calculated using the interpolation-extrapolation method detailed in ~\ref(sec:methods). To determine the reliability of this prediction method, SWING used
the Python library "scikit-learn" to calculate the R-squared value when predicting known data. This R-squared value was calculated as 0.4980, indicating that 49.80\% of the 
variability in the unknown coefficients is explained by SWING's predicition model.

\begin{table}
\centering
\caption{Estimated Optical Coefficients}
\label{CoefficientTable}
\setlength{\tabcolsep}{3pt}
\renewcommand{\arraystretch}{1.5}
\resizebox{\columnwidth}{!}{%
\begin{tabular}{|L{45pt}|L{60pt}|C{60pt}|C{60pt}|}
\hline
Tissue & Wavelength, nm & Absorption Coefficient $\mu_{a}$, cm$^{-1}$ &  Reduced Scattering Coefficient $\mu_{s}^{'}$, cm$^{-1}$ \\
\hline
Scalp & 810 & 0.505 & 2.35 \\
&       980 & 1.23 & 2.35 \\
&       1064 & 1.23 & 2.35 \\
&       1550 & 1.23 & 2.35 \\
Skull & 810 & 0.099 & 2.35 \\
&       980 & 1.23 & 2.35 \\
&       1064 & 1.23 & 2.35 \\
&       1550 & 1.23 & 2.35 \\
Gray Matter & 810 & 0.455 & 2.35 \\
&       980 & 1.23 & 2.35 \\
&       1064 & 1.23 & 2.35 \\
&       1550 & 1.23 & 2.35 \\
White Matter & 810 & 1.23 & 2.35 \\
&       980 & 1.23 & 2.35 \\
&       1064 & 1.23 & 2.35 \\
&       1550 & 1.23 & 2.35 \\
\hline
\end{tabular}%
}
\label{tab1}
\end{table}

\begin{figure}[!htb]
    \center{\includegraphics[width=\columnwidth]
    {Figures/Fluence_Distribution_810nm_CZ.png}}
    \caption{\label{fig:810-CZ} 810 nm CZ Position}
\end{figure}

\begin{figure}[!htb]
    \center{\includegraphics[width=\columnwidth]
    {Figures/Fluence_Distribution_810nm_45deg.png}}
    \caption{\label{fig:810-45} 810 nm 45 Degree Position}
\end{figure}

\begin{figure}[!htb]
    \center{\includegraphics[width=\columnwidth]
    {Figures/Fluence_Distribution_810nm_Cochlear.png}}
    \caption{\label{fig:810-Cochlear} 810 nm Cochlear Position}
\end{figure}

\begin{figure}[!htb]
    \center{\includegraphics[width=\columnwidth]
    {Figures/Fluence_Distribution_810nm_Intranasal.png}}
    \caption{\label{fig:810-Intra} 810 nm Intranasal Position}
\end{figure}

\begin{figure}[!htb]
    \center{\includegraphics[width=\columnwidth]
    {Figures/Fluence_Distribution_980nm_CZ.png}}
    \caption{\label{fig:980-CZ} 980 nm CZ Position}
\end{figure}

\begin{figure}[!htb]
    \center{\includegraphics[width=\columnwidth]
    {Figures/Fluence_Distribution_980nm_45deg.png}}
    \caption{\label{fig:980-45} 980 nm 45 Degree Position}
\end{figure}

\begin{figure}[!htb]
    \center{\includegraphics[width=\columnwidth]
    {Figures/Fluence_Distribution_980nm_Cochlear.png}}
    \caption{\label{fig:980-Cochlear} 980 nm Cochlear Position}
\end{figure}

\begin{figure}[!htb]
    \center{\includegraphics[width=\columnwidth]
    {Figures/Fluence_Distribution_980nm_Intranasal.png}}
    \caption{\label{fig:980-Intra} 980 nm Intranasal Position}
\end{figure}

\begin{figure}[!htb]
    \center{\includegraphics[width=\columnwidth]
    {Figures/Fluence_Distribution_1064nm_CZ.png}}
    \caption{\label{fig:1064-CZ} 1064 nm CZ Position}
\end{figure}

\begin{figure}[!htb]
    \center{\includegraphics[width=\columnwidth]
    {Figures/Fluence_Distribution_1064nm_45deg.png}}
    \caption{\label{fig:1064-45} 1064 nm 45 Degree Position}
\end{figure}

\begin{figure}[!htb]
    \center{\includegraphics[width=\columnwidth]
    {Figures/Fluence_Distribution_1064nm_Cochlear.png}}
    \caption{\label{fig:1064-Cochlear} 1064 nm Cochlear Position}
\end{figure}

\begin{figure}[!htb]
    \center{\includegraphics[width=\columnwidth]
    {Figures/Fluence_Distribution_1064nm_Intranasal.png}}
    \caption{\label{fig:1064-Intra} 1064 nm Intranasal Position}
\end{figure}

\begin{figure}[!htb]
    \center{\includegraphics[width=\columnwidth]
    {Figures/Fluence_Distribution_1550nm_CZ.png}}
    \caption{\label{fig:1550-CZ} 1550 nm CZ Position}
\end{figure}

\begin{figure}[!htb]
    \center{\includegraphics[width=\columnwidth]
    {Figures/Fluence_Distribution_1550nm_45deg.png}}
    \caption{\label{fig:1550-45} 1550 nm 45 Degree Position}
\end{figure}

\begin{figure}[!htb]
    \center{\includegraphics[width=\columnwidth]
    {Figures/Fluence_Distribution_1550nm_Cochlear.png}}
    \caption{\label{fig:1550-Cochlear} 1550 nm Cochlear Position}
\end{figure}

\begin{figure}[!htb]
    \center{\includegraphics[width=\columnwidth]
    {Figures/Fluence_Distribution_1550nm_Intranasal.png}}
    \caption{\label{fig:1550-Intra} 1550 nm Intranasal Position}
\end{figure}

\section{Discussion}
\label{sec:next steps}

\section{Conclusion}
\label{sec:conclusion}

\section*{Acknowledgment}

\section*{References}

\end{document}
